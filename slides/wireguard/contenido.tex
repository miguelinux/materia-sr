% ex: ts=2 sw=2 sts=2 et filetype=tex
% SPDX-License-Identifier: CC-BY-SA-4.0

\section{Introducción a Wireguard}

\begin{frame}[c]{¿Qué es Wireguard?}
    \begin{columns}
        \column{0.5\textwidth}
        WireGuard es una aplicación de software libre y de código abierto
        y un protocolo de comunicación que implementa técnicas de red
        privada virtual (\textbf{VPN}) para crear \emph{conexiones seguras
        punto a punto} en configuraciones enrutadas o puenteadas.
        \column{0.5\textwidth}
        \begin{center}
            \includegraphics[scale=0.5]{wireguard-logo.png}
        \end{center}
    \end{columns}
\end{frame}

\begin{frame}[c]{¿Qué es Wireguard? (continuación)}
    \begin{columns}
        \column{0.5\textwidth}
        \begin{center}
            \includegraphics[scale=0.5]{kr-donenfeld.jpg}
        \end{center}
        \column{0.5\textwidth}
        Se ejecuta como un módulo dentro del núcleo \textbf{Linux} y tiene
        como objetivo un mejor rendimiento que los protocolos de tunelización
        IPsec y OpenVPN.

        \vspace{\baselineskip}
        Fue escrito por \textbf{Jason A. Donenfeld} y se publica bajo la
        segunda versión de la GNU General Public License (GPL).
    \end{columns}
\end{frame}

\begin{frame}[c]{Características}
  \begin{itemize}
    \item Tiene como objetivo proporcionar una VPN que sea
          simple y altamente efectiva.
    \pausa
    \item Código fuente de alrededor de 4000 líneas.
    \pausa
    \item Utiliza criptografía de última generación, como el \emph{framework}
          del protocolo Noise, Curve25519, ChaCha20, Poly1305, BLAKE2,
          SipHash24, HKDF y construcciones seguras y confiables.

  \end{itemize}
\end{frame}

\section{Creación de llaves}

\begin{frame}[fragile]
  \frametitle{Generando llave privada}

  WireGuard requiere claves públicas y privadas codificadas en base64.
  Estos se pueden generar utilizando la utilidad \textbf{wg(8)}: 

  \vspace{\baselineskip}
  \begin{lstlisting}[language=Bash]
  $ umask 077
  $ wg genkey > privatekey
  \end{lstlisting}

  \vspace{\baselineskip}
  esto creará una clave privada en la salida estándar que contiene una
  nueva clave privada. 
\end{frame}

\begin{frame}[fragile]
  \frametitle{Generando llave publica}

  Luego puede derivar su clave pública de su clave privada:

  \vspace{\baselineskip}
  \begin{lstlisting}[language=bash]
  $ wg pubkey < privatekey > publickey
  $
  \end{lstlisting}

  \vspace{\baselineskip}
  Esto leerá la clave privada de la entrada estándar y escribirá la
  clave pública correspondiente en la clave pública en la salida estándar.
\end{frame}

\begin{frame}[fragile]
  \frametitle{Generando ambas llaves a la vez}

  Por supuesto, puedes hacer todo esto a la vez: 

  \vspace{\baselineskip}
  \begin{lstlisting}[language=Bash]
  $ umask 077
  $ wg genkey | tee privatekey | wg pubkey > publickey
  \end{lstlisting}

\end{frame}

\section{Configurando las interfaces de red}

\begin{frame}[fragile]
  \frametitle{Agregando una interfaz}

  Se puede agregar una nueva interfaz a través de \textbf{ip-link(8)},
  que debería manejar automáticamente la carga del módulo:

  \vspace{\baselineskip}
  \begin{lstlisting}[language=Bash]
  # ip link add dev wg0 type wireguard
  \end{lstlisting}

  Los usuarios que no son de Linux escribirán en su lugar wireguard-go wg0.
\end{frame}

\begin{frame}[fragile]
  \frametitle{Asignando IP}

  Se puede asignar una dirección IP y un "peer" con \textbf{ip-address(8)}

  \vspace{\baselineskip}
  \begin{lstlisting}[language=Bash]
  # ip address add dev wg0 192.168.2.1/24
  \end{lstlisting}

  \vspace{\baselineskip}
  O, si solo hay dos pares en total, algo como esto podría ser más deseable:

  \vspace{\baselineskip}
  \begin{lstlisting}[language=Bash]
  # ip address add dev wg0 192.168.2.1 peer 192.168.2.2
  \end{lstlisting}
\end{frame}

\begin{frame}[fragile]
  \frametitle{}

  La interfaz se puede configurar con claves y puntos finales del
  mismo nivel con la utilidad \textbf{wg(8)} incluida: 

  \vspace{\baselineskip}
  \begin{lstlisting}[language=Bash]
  # wg setconf wg0 myconfig.conf
  \end{lstlisting}

  \vspace{\baselineskip}
  o con

  \vspace{\baselineskip}
  \begin{lstlisting}[language=Bash]
  # wg set wg0 listen-port 51820 private-key /path/to/private-key peer ABCDEF... allowed-ips 192.168.88.0/24 endpoint 209.202.254.14:8172
  \end{lstlisting}
\end{frame}

\begin{frame}[fragile]
  \frametitle{}

  Finalmente, la interfaz se puede activar con  \textbf{ip-link(8)}: 

  \vspace{\baselineskip}
  \begin{lstlisting}[language=Bash]
  # ip link set up dev wg0
  \end{lstlisting}

  \vspace{\baselineskip}
  También existen los comandos \textbf{wg show} y \textbf{wg showconf}, para ver
  la configuración actual. Llamar a wg sin argumentos por defecto es 
  lamar a wg show en todas las interfaces de WireGuard.

\end{frame}

%\begin{frame}[c]{Caracteres de escape}
%\end{frame}

%\begin{frame}[fragile]
%  \frametitle{Cadenas de texto}
%
%  \vspace{\baselineskip}
%  \textcolor{codeKeyword}{print}():
%
%  \vspace{\baselineskip}
%  \begin{lstlisting}[language=Python]
%  \end{lstlisting}
%\end{frame}
