% ex: ts=2 sw=2 sts=2 et filetype=tex
% SPDX-License-Identifier: CC-BY-SA-4.0
\documentclass[aspectratio=169]{beamer}

\usetheme{Boadilla}

\setbeamertemplate{navigation symbols}{} % To remove the navigation symbols from the bottom of all slides

\graphicspath{{../img/}{../styles/iteso/}}
% use the command "usepackage" instead of "usecolortheme" to use the path to styles
\usepackage{../styles/iteso/beamercolorthemeiteso}

\usepackage[utf8]{inputenc}
\usepackage[T1]{fontenc} %% https://tex.stackexchange.com/questions/664/why-should-i-use-usepackaget1fontenc
\usepackage{graphicx} % Allows including images
\usepackage{listings}
\usepackage{colortbl}

\title{Wireguard} % The short title appears at the bottom of every slide,
                                               % the full title is only on the title page
\author{Miguel Bernal Marin}
\institute[ITESO]
{
 ITESO, Universidad \\
 Jesuita de Guadalajara \\
\medskip
\textit{miguel.bernal@iteso.mx}
}
\date{
    9 de Febrero de 2022
} % Date, can be changed to a custom date

%New colors defined below
\definecolor{codeBakground}{rgb}{0.95,0.95,0.92}
\definecolor{codeComment}{rgb}{0.0, 0.0, 0.0}
\definecolor{codeKeyword}{rgb}{0.647, 0.165, 0.165}
\definecolor{codeKeyword2}{rgb}{0.0, 0.545,0.545}
\definecolor{codeNumbers}{rgb}{0.5,0.5,0.5}
\definecolor{codeString}{rgb}{1.0, 0.0, 1.0}
\definecolor{light-gray}{gray}{0.90}

%Code listing style named "mystyle"
\lstdefinestyle{mystyle}{
  backgroundcolor=\color{codeBakground},
  commentstyle=\color{codeComment},
  keywordstyle=\color{codeKeyword},
  keywordstyle={[2]\color{codeKeyword2}}, % Built-in
  numberstyle=\tiny\color{codeNumbers},
  stringstyle=\color{codeString},
  basicstyle=\ttfamily\footnotesize,
  breakatwhitespace=false,
  breaklines=true,
  captionpos=b,
  keepspaces=true,
  numbers=left,
  numbersep=5pt,
  showspaces=false,
  showstringspaces=false,
  showtabs=false,
  upquote=true,                      % requires textcomp
  tabsize=2
}

%"mystyle" code listing set
\lstset{style=mystyle}

%------------------------------------------------------------
%The next block of commands puts the table of contents at the 
%beginning of each section and highlights the current section:
\AtBeginSection[]
{
  \begin{frame}
    \frametitle{Contenido}
    \tableofcontents[currentsection]
  \end{frame}
}
%------------------------------------------------------------

\newcommand{\pausa}{\pause} % Para usar una pausa en las presentaciones
%\newcommand{\pausa}{}      % Para que NO salgan las pausas

\begin{document}
{ % Portada de la universidad
  \setbeamertemplate{footline}{}
  \usebackgroundtemplate{\includegraphics[width=\paperwidth]{fondo-portada.png}}
  \begin{frame}
  \end{frame}
}

\begin{frame}
    \titlepage
\end{frame}

\usebackgroundtemplate{\includegraphics[width=\paperwidth]{fondo.png}}

\begin{frame}
    \frametitle{Contenido}
    \tableofcontents
\end{frame}

% ex: ts=2 sw=2 sts=2 et filetype=tex
% SPDX-License-Identifier: CC-BY-SA-4.0

\section{Instalación de VirtualBox}

\begin{frame}[c]{Pantalla de Bienvendia}
  \begin{center}
    \includegraphics[scale=0.6]{01.png}
  \end{center}
\end{frame}

\begin{frame}[c]{Selección de paquetes a instalar}
  \begin{center}
    \includegraphics[scale=0.6]{02.png}
  \end{center}
\end{frame}

\begin{frame}[c]{Selección de características a instalar}
  \begin{center}
    \includegraphics[scale=0.6]{03.png}
  \end{center}
\end{frame}

\begin{frame}[c]{Interfaces de red}
  \begin{center}
    \includegraphics[scale=0.6]{04.png}
  \end{center}
\end{frame}

\begin{frame}[c]{Listos para instalar}
  \begin{center}
    \includegraphics[scale=0.6]{05.png}
  \end{center}
\end{frame}

\begin{frame}[c]{Permisos para instalar}
  \begin{center}
    \includegraphics[scale=0.6]{06.png}
  \end{center}
\end{frame}

\begin{frame}[c]{Terminar instalación}
  \begin{center}
    \includegraphics[scale=0.6]{07.png}
  \end{center}
\end{frame}


\section{Creación de una maquina virtual}

\begin{frame}[c]{Pantalla principal}
  \begin{center}
    \includegraphics[scale=0.6]{08.png}
  \end{center}
\end{frame}

\begin{frame}[c]{Nombre de la VM y SO}
  \begin{center}
    \includegraphics[scale=0.6]{09.png}
  \end{center}
\end{frame}

\begin{frame}[c]{Versión de Linux}
  \begin{center}
    \includegraphics[scale=0.6]{10.png}
  \end{center}
\end{frame}

\begin{frame}[c]{Tamaño de la memoria}
  \begin{center}
    \includegraphics[scale=0.6]{11.png}
  \end{center}
\end{frame}

\begin{frame}[c]{Creación de disco duro}
  \begin{center}
    \includegraphics[scale=0.6]{12.png}
  \end{center}
\end{frame}

\begin{frame}[c]{Tipo de disco duro}
  \begin{center}
    \includegraphics[scale=0.6]{13.png}
  \end{center}
\end{frame}

\begin{frame}[c]{Archivo dinámico de disco duro}
  \begin{center}
    \includegraphics[scale=0.6]{14.png}
  \end{center}
\end{frame}

\begin{frame}[c]{Tamaño de disco duro}
  \begin{center}
    \includegraphics[scale=0.6]{15.png}
  \end{center}
\end{frame}

\begin{frame}[c]{Pantalla principal}
  \begin{center}
    \includegraphics[scale=0.6]{16.png}
  \end{center}
\end{frame}

\section{Configurando una maquina virtual}

\begin{frame}[c]{Ventana de configuración}
  \begin{center}
    \includegraphics[scale=0.5]{17.png}
  \end{center}
\end{frame}

\begin{frame}[c]{Almacenamiento}
  \begin{center}
    \includegraphics[scale=0.5]{18.png}
  \end{center}
\end{frame}

\begin{frame}[c]{Seleccionar ISO a instalar}
  \begin{center}
    \includegraphics[scale=0.5]{19.png}
  \end{center}
\end{frame}

\begin{frame}[c]{Verificar ISO}
  \begin{center}
    \includegraphics[scale=0.5]{20.png}
  \end{center}
\end{frame}

\begin{frame}[c]{Habilitar EFI y orden de arranque}
  \begin{center}
    \includegraphics[scale=0.5]{21.png}
  \end{center}
\end{frame}

\begin{frame}[c]{Apagando una VM a la fuerza}
  \begin{center}
    \includegraphics[scale=0.6]{22.png}
  \end{center}
\end{frame}

\section{Instalando Centos Stream 9}

\begin{frame}[c]{Sistema de Arranque (bootloader) GRUB}
  \begin{center}
    \includegraphics[scale=0.24]{cs9-01.png}
  \end{center}
\end{frame}

\begin{frame}[c]{Selección de Idioma}
  \begin{center}
    \includegraphics[scale=0.24]{cs9-02.png}
  \end{center}
\end{frame}

\begin{frame}[c]{Pantalla de configuración}
  \begin{center}
    \includegraphics[scale=0.24]{cs9-03.png}
  \end{center}
\end{frame}

\begin{frame}[c]{Selección de Software}
  \begin{center}
    \includegraphics[scale=0.24]{cs9-04.png}
  \end{center}
\end{frame}

\begin{frame}[c]{Destino de la instalación}
  \begin{center}
    \includegraphics[scale=0.24]{cs9-05.png}
  \end{center}
\end{frame}

\begin{frame}[c]{Configuración de almacenamiento personaliza}
  \begin{center}
    \includegraphics[scale=0.24]{cs9-06.png}
  \end{center}
\end{frame}

\begin{frame}[c]{Particionado Manual}
  \begin{center}
    \includegraphics[scale=0.24]{cs9-07.png}
  \end{center}
\end{frame}

\begin{frame}[c]{Creación automática de particiones}
  \begin{center}
    \includegraphics[scale=0.24]{cs9-08.png}
  \end{center}
\end{frame}

\begin{frame}[c]{Resumen de cambios}
  \begin{center}
    \includegraphics[scale=0.24]{cs9-09.png}
  \end{center}
\end{frame}

\begin{frame}[c]{Contraseña de root}
  \begin{center}
    \includegraphics[scale=0.24]{cs9-10.png}
  \end{center}
\end{frame}

\begin{frame}[c]{Creación de usuario}
  \begin{center}
    \includegraphics[scale=0.24]{cs9-11.png}
  \end{center}
\end{frame}

\begin{frame}[c]{Comenzar la instalación}
  \begin{center}
    \includegraphics[scale=0.24]{cs9-12.png}
  \end{center}
\end{frame}

\begin{frame}[c]{Progreso de la instalación}
  \begin{center}
    \includegraphics[scale=0.24]{cs9-13.png}
  \end{center}
\end{frame}

\begin{frame}[c]{Reinicio del sistema}
  \begin{center}
    \includegraphics[scale=0.24]{cs9-14.png}
  \end{center}
\end{frame}

\begin{frame}[c]{Iniciando SO}
  \begin{center}
    \includegraphics[scale=0.24]{cs9-15.png}
  \end{center}
\end{frame}

\begin{frame}[c]{Ingreso al sistema}
  \begin{center}
    \includegraphics[scale=0.24]{cs9-16.png}
  \end{center}
\end{frame}

\section{Instalando Ubuntu Server}

\begin{frame}[c]{Sistema de Arranque (bootloader) GRUB}
  \begin{center}
    \includegraphics[scale=0.24]{us-01.png}
  \end{center}
\end{frame}

\begin{frame}[c]{Selección de Idioma}
  \begin{center}
    \includegraphics[scale=0.24]{us-02.png}
  \end{center}
\end{frame}

\begin{frame}[c]{Actualizar el instalador}
  \begin{center}
    \includegraphics[scale=0.24]{us-03.png}
  \end{center}
\end{frame}

\begin{frame}[c]{Configuración del teclado}
  \begin{center}
    \includegraphics[scale=0.24]{us-04.png}
  \end{center}
\end{frame}

\begin{frame}[c]{Configuración de red}
  \begin{center}
    \includegraphics[scale=0.24]{us-05.png}
  \end{center}
\end{frame}

\begin{frame}[c]{Configuración de proxy}
  \begin{center}
    \includegraphics[scale=0.24]{us-06.png}
  \end{center}
\end{frame}

\begin{frame}[c]{Configuración de repositorio de Ubuntu}
  \begin{center}
    \includegraphics[scale=0.24]{us-07.png}
  \end{center}
\end{frame}

\begin{frame}[c]{Selección de disco de instalación}
  \begin{center}
    \includegraphics[scale=0.24]{us-08.png}
  \end{center}
\end{frame}

\begin{frame}[c]{Configuración de particiones}
  \begin{center}
    \includegraphics[scale=0.24]{us-09.png}
  \end{center}
\end{frame}

\begin{frame}[c]{Confirmación de la creación de particiones}
  \begin{center}
    \includegraphics[scale=0.24]{us-10.png}
  \end{center}
\end{frame}

\begin{frame}[c]{Configuración de usuario}
  \begin{center}
    \includegraphics[scale=0.24]{us-11.png}
  \end{center}
\end{frame}

\begin{frame}[c]{Instalar servidor de SSH}
  \begin{center}
    \includegraphics[scale=0.24]{us-12.png}
  \end{center}
\end{frame}

\begin{frame}[c]{Instalar otros servicios}
  \begin{center}
    \includegraphics[scale=0.24]{us-13.png}
  \end{center}
\end{frame}

\begin{frame}[c]{Instalación y actualización}
  \begin{center}
    \includegraphics[scale=0.24]{us-14.png}
  \end{center}
\end{frame}

\begin{frame}[c]{Reinicio del sistema}
  \begin{center}
    \includegraphics[scale=0.24]{us-15.png}
  \end{center}
\end{frame}

\begin{frame}[c]{Ingreso al sistema}
  \begin{center}
    \includegraphics[scale=0.24]{us-16.png}
  \end{center}
\end{frame}

\section{Instalando Kali Linux}

\begin{frame}[c]{Sistema de Arranque (bootloader) GRUB}
  \begin{center}
    \includegraphics[scale=0.24]{kl-01.png}
  \end{center}
\end{frame}

\begin{frame}[c]{Configuración de idioma}
  \begin{center}
    \includegraphics[scale=0.24]{kl-02.png}
  \end{center}
\end{frame}

\begin{frame}[c]{Selección de Ubicación}
  \begin{center}
    \includegraphics[scale=0.24]{kl-03.png}
  \end{center}
\end{frame}

\begin{frame}[c]{Configuración de teclado}
  \begin{center}
    \includegraphics[scale=0.24]{kl-04.png}
  \end{center}
\end{frame}

\begin{frame}[c]{configuración de red}
  \begin{center}
    \includegraphics[scale=0.24]{kl-05.png}
  \end{center}
\end{frame}

\begin{frame}[c]{Nombre de dominio}
  \begin{center}
    \includegraphics[scale=0.24]{kl-06.png}
  \end{center}
\end{frame}

\begin{frame}[c]{Nombre de usuario real}
  \begin{center}
    \includegraphics[scale=0.24]{kl-07.png}
  \end{center}
\end{frame}

\begin{frame}[c]{Nombre de usuario del sistema}
  \begin{center}
    \includegraphics[scale=0.24]{kl-08.png}
  \end{center}
\end{frame}

\begin{frame}[c]{Contraseña}
  \begin{center}
    \includegraphics[scale=0.24]{kl-09.png}
  \end{center}
\end{frame}

\begin{frame}[c]{Configurar Zona horaria}
  \begin{center}
    \includegraphics[scale=0.24]{kl-10.png}
  \end{center}
\end{frame}

\begin{frame}[c]{Partición amiento de disco duro}
  \begin{center}
    \includegraphics[scale=0.24]{kl-11.png}
  \end{center}
\end{frame}

\begin{frame}[c]{Selección de disco duro}
  \begin{center}
    \includegraphics[scale=0.24]{kl-12.png}
  \end{center}
\end{frame}

\begin{frame}[c]{Esquema de partición}
  \begin{center}
    \includegraphics[scale=0.24]{kl-13.png}
  \end{center}
\end{frame}

\begin{frame}[c]{Particiones}
  \begin{center}
    \includegraphics[scale=0.24]{kl-14.png}
  \end{center}
\end{frame}

\begin{frame}[c]{Confirmar particionamiento}
  \begin{center}
    \includegraphics[scale=0.24]{kl-15.png}
  \end{center}
\end{frame}

\begin{frame}[c]{Instalación sistema base}
  \begin{center}
    \includegraphics[scale=0.24]{kl-16.png}
  \end{center}
\end{frame}

\begin{frame}[c]{Seleccionar paquetes}
  \begin{center}
    \includegraphics[scale=0.24]{kl-17.png}
  \end{center}
\end{frame}

\begin{frame}[c]{Instalación de los paquetes extras}
  \begin{center}
    \includegraphics[scale=0.24]{kl-18.png}
  \end{center}
\end{frame}



{ % Portada de la universidad
  \setbeamertemplate{footline}{}
  \usebackgroundtemplate{\includegraphics[width=\paperwidth]{fondo-portada.png}}
  \begin{frame}
  \end{frame}
}
\end{document}
